Ο στόχος της παρούσας πτυχιακής διατριβής ήταν να εξετάσει την δυναμική αλληλεπίδραση ενός γιγάντιου πλανήτη με έναν πρωτοπλανητικό δίσκο σκόνης, να εξακριβώσει και τελικά να αναδείξει κάποιον <<δείκτη>> που θα αποτελεί την <<υπογραφή>> ύπαρξης του πλανήτη στο σύστημα. Επείδη η ύπαρξη τέτοιων δίσκων γίνεται αντιληπτή μέσω της ισχυρής {\en far-IR/submillimeter} ακτινοβολίας που εκπέμπουν, σαν στόχος ανίχνευσης τέτοιων δεικτών τέθηκε η δημιουργία  απεικόνισεων του δίσκου με τις διακριτικές ικανότητες της {\en ALMA}. Σκοπός των απεικονίσεων ήταν η ανάδειξη του <<δείκτη>> ύπαρξης του πλανήτη, η ποιοτική ανάλυση των εικόνων και εκλογή συμπερασμάτων σχετικά με την πραγματική απεικόνιση τέτοιων δίσκων.\\

Βασικό στοιχείο στην μελέτη μας ήταν η δυναμική εξέλιξη του συστήματος για χρόνικό διάστημα που αντιστοιχεί σε πολλές περιόδους περιφοράς του πλανήτη, ώστε τα αποτελέσματα που θα πάρουμε να είναι ασφαλή. Απο την σκοπιά της {\it δυναμικής ανάλυσης} του προβλήματος είδαμε ότι η παρουσία του πλανήτη επιβεβαιώνεται όχι μόνο απο τον <<κενό>> δακτύλιο, τον οποίο δημιουργεί στο επίπεδο της τροχιάς του, αλλα και απο την δημιουργία των δύο ομάδων (<<Τρωικοί>>) με τους οποίους βρίσκεται σε σταθερό συντονισμό $1:1$ και μοιράζεται την τροχιά του. Είδαμε ότι το εύρος του δακτυλίου εξαρτάται απο την {\it ακτίνα {\en Hill}} του πλανήτη και κατ' επέκταση απο τον λόγο της μάζας του πλανήτη προς την μάζα του αστέρα του συστήματος. Ένα ακόμα σημαντικό συμπέρασμα είναι ότι η  ύπαρξη του πλανήτη μπορεί να ιχνηλατηθεί και σε αποστάσεις αρκετά μεγαλύτερες απο την {\it ακτίνα {\en Hill}} του$\cdot$ μέσω της εμφάνισης συντονισμών μέσης κίνησης. Στην περίπτωση αυτή παρατηρούνται τοπικά ελάχιστα στην αριθμητική πυκνότητα των σωματιδίων της σκόνης σε αποστάσεις που εμπίπτουν στους ασταθείς αυτούς συντονισμούς, καθώς τα στοιχεία της τροχιάς των σωματιδίων μεταβάλλονται έντονα και σε σύντομα χρονικά διαστήματα.\\
Απο την σκοπιά της ανίχνευσης της ηλεκτρομαγνητικής ακτινοβολίας του δίσκου στο {\en far-IR/submillimeter} και τελικά όπως αυτή θα φαινόταν μετά απο χαρτογράφιση με την {\en ALMA} είδαμε ότι ο <<δείκτης>> που επιβεβαιώνει την ύπαρξη του πλανήτη και είναι ανιχνεύσιμος είναι ο <<κενός>> δακτύλιος που δημιουργεί. Στην περίπτωση όμως αυτή, καταλάβαμε ότι η ανιχνευσιμότητα του παραπάνω <<δείκτη>> εξαρτάται απο πολλούς παράγοντες που θα μπορούσαν να καταταχθούν συνοπτικα σε δύο κατηγορίες: 

\begin{enumerate}

\item Φυσικά χαρακτηριστικά του συστήματος Δίσκου-Αστέρα

Αρχικά είδαμε πόσο σημαντική είναι η εκλογή του μοντέλου ακτινοβολίας$\cdot$ καθώς ο πρώτος παράγοντας που καθορίζει τη θέρμοκρασία της σκόνης είναι η βολομετρική ροή του αστέρα. Ταυτόχρονα, ο δεύτερος παράγοντας είναι τα φυσικά χαρακτηριστικά της σκόνης που συγκροτούν τον δίσκο, πιο συγκεκριμένα η σύσταση, το σχήμα και το μέγεθος, αφού προσδιορίζουν όχι μόνο πόση ενέργεια θα απορροφήσει και θα επανεκπέμψει η σκόνη αλλά και σε ποιά μήκη κύματος$\cdot$ ορίζοντας τελικά την θερμοκρασίας ισορροπίας της. Απο την πλευρά της, η τελική θερμοκρασία του δίσκου σε συνδυασμό με την επιφανειακή του πυκνότητα και την απόσταση του καθορίζουν στην τελική ποσότητα μονοχρωματικής ροής που θα φτάσει σε αυτά.

\item Χαρακτηριστικά-Δυνατότητες των οργάνων παρατήρησης

Στην δεύτερη αυτή κατηγορία είδαμε την ποσοτικοποίηση των περιορισμών της διακριτικής ικανότητας των σύγχρονων συμβολομέτρων. Πιο συγκεκριμένα, στα μήκη κύματος που ο δίσκος εκπέμπει αποτελεσματικά σε συνδυασμό με τις τιμές διακριτικής ικανότητας που μπορούν επιτύχουν στις αντίστοιχες συχνότητες. Είδαμε την σημασία των παραπάνω στην περίπτωση ανίχνευσης του <<κενού>> δακτυλίου για το μοντέλο μας, όπου η σχέση μεταξύ διακριτικής ικανότητας και συχνότητας ανίχνευσης ουσιαστικά καθιστούσε το αν ο δείκτης θα ήταν ανιχνεύσιμος σε κάποια απεικόνιση ή όχι.
\end{enumerate}

Στην δημιουργία και μελέτη προσομοιώσεων συστημάτων εξίσου βασικό με την εκλογή συμπερασμάτων είναι και η ικανότητα προσδιορισμού των σημείων όπου το μοντέλο εισάγει περιορισμούς λόγω παροδοχών. Η κατανόηση των περιορισμών αυτών είναι πολύ σημαντική καθώς χαράζει τον δρόμο στον οποίο πρέπει να κινηθούμε βελτιώνοντας το μοντέλο μας και τελικά παράγοντας αποτελέσματα πιο κοντά στον φυσικό κόσμο.\\ 

Στην περίπτωση του μοντέλου μας, όπως αναφέρθηκε, τα {\en test particles} αντιμετωπίζονται ως σωματίδια μηδενικής μάζας απο τον {\en SWIFT}, δεν αλληλεπιδρούν μεταξύ τους και δεν επηρέαζουν βαρυτικά τον πλανήτη. Η παραπάνω συμπεριφορά φυσικά εισάγει ένα σφάλμα στις τελικές τιμές των διανυσμάτων θέσης και ορμής που λαμβανούμε για τα {\en test particles} μετα την ολοκλήρωση των $10Myrs$. Το σφάλμα όμως αυτό δεν επηρεάζει σημαντικά τα αποτελέσματα μας, διότι η αλλαγή των διανυσμάτων θέσης και ορμής των {\en test particles} την χρονική στιγμή $t_1=10Myrs$ δεν αλλάζει το γεγονός της δημιουργίας του κενού δακτυλίου στην τροχιά του πλανήτη. Ακόμα η μάζα του πλανήτη εξακολουθεί να είναι μεγαλυτερη απο αυτή των {\en test particles} κατα 6 τάξεις μεγέθους άρα παραμένει η αντίληψη ότι αυτός επηρεάζει σημαντικά τα {\en test particles} και επηρεάζεται λιγότερο απο αυτά. Ακόμα η παραδοχή του μικρού οπτικού βάθους για τον δίσκο εισάγει, όπως περιγράψαμε, ένα σφάλμα στον υπολογισμό της κατανομής θερμοκρασιών του δίσκου το οποίο έχει ως αποτέλεσμα να λαμβάνουμε τις μεγαλύτερες τιμές ειδικών εντάσεων της ακτινοβολίας απο τον δίσκο. Απο αυτή την άποψη, μαζί με την υπόθεση μεγάλων ποσοτήτων μεσοαστρικής σκόνης για τον δίσκο τα αποτελέσματα μας είναι μαξιμαλιστικά.\\

Συνοψίζοντας, η πραγμάτωση παραδοχών για την δημιουργία προσομοιώσεων του δίσκου είναι αναγκαία$\cdot$ καθώς κανένας κώδικας δεν μπορεί να διαχειριστεί την πολυπλοκότητα τέτοιων συστημάτων ούτε τον αριθμό των σωματιδίων του με πλήρη ταυτίση της πραγματικότητας! Σαν αποτέλεσμα είναι λογικό η υπολογιστική ακρίβεια των αποτελεσμάτων να αποκλίνει σε κάποιο βαθμό απο την πραγματική εικόνα, {\it η ποιοτική όμως συμπεριφόρα του συστήματος μπορεί μελετηθεί με πολύ καλή ακρίβεια δίνοντας μας την δυνατότητα εκλογής συμπερασμάτων για την φυσική συμπεριφορά αυτού}. Το <<στοίχημα>> για τα επόμενα χρόνια είναι η δημιουργία αλγορίθμων που θα μπορούν να διαχειριστούν μεγαλύτερες τιμές σωματιδίων βρίσκοντας τις βαρυτικές δυνάμεις που αναπτύσσονται μεταξύ όλων των σωμάτων του συστήματος. Ακόμα η δημιουργία αλγορίθμων που θα μελετάνε αναλυτικά το πρόβλημα της διάδοσης ακτινοβολίας και της αλληλεπίδρασης της με σωματίδια διαφορετικών μεγεθών, σχημάτων και σύστασης$\cdot$ καθορίζοντας τελικά το προφίλ κατανομής θερμοκρασιών σε διαφορετικούς δίσκους σκόνης με μεγαλύτερη ακρίβεια.
