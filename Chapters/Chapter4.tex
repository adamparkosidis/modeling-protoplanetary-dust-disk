 Προκειμένου να υπολογίσουμε την {\it Θερμοκρασία Ισορροπίας} των σωματιδίων της σκόνης στον δίσκο πρέπει αρχικά να ορίσουμε τα χαρακτηριστικά του πεδίου της ακτινοβολίας. 

\section{Επιλογή Πεδίου Ακτινοβολίας}

Στο σημείο αυτό ας θεωρήσουμε ως πεδίο ακτινοβολίας ένα αστέρι και την γύρω περιοχή του. Φυσικά δε μπορούμε να έχουμε αυστηρά συνθήκες θερμοδυναμικής ισορροπίας καθώς οι φυσικές παράμετροι, όπως η θερμοκρασία, μεταβάλλονται στον χώρο και στον χρόνο. Μπορούμε όμως να υποθέσουμε συνθήκες {\it τοπικής θερμοδυναμικής Ισορροπίας} για μεταβολές αρκετά μικρές, σε κλίμακες συγκρίσιμες με την μέση ελεύθερη διαδρομή των φωτονίων. Έτσι υποθέτουμε ότι {\it η επιφάνεια του αστέρα εκπέμπει ακτινοβολία σαν μέλαν σώμα} \\

Στην παρούσα εργασία επιλέξαμε σαν κεντρικό άστρο τον Ήλιο. Θεωρώντας σφαιρική συμμετρία και απο την σχέση \eqref{eq:TotalPossitiveFlux} η συνολική θετική ροή που εκπέμπεται από την επιφάνεια του προς όλες τις κατευθύνσεις δίνεται απο την σχέση \eqref{eq:TotalPossitiveFlux}:

\begin{equation}\label{eq:SunsTotalPosFLux}
  F^{+} =\sigma T_{e}^4, \text{όπου $T_{e}$ η ενεργός θερμοκρασία της επιφάνεις του Ήλιου}
\end{equation}  

Η συνολική ηλεκτρομαγνητική ισχύς ή αντίστοιχα η ενέργεια ανα μονάδα χρόνου που εκπέμπει ο Ήλιος απο την επιφάνεια του σε όλο το ηλεκτρομαγνητικό φάσμα ονομάζεται {\it Φωτεινότητα}, $L_\odot$ και προφανώς: 

\begin{equation}\label{eq:Brightness}
  L_\odot = 4\pi R_{\odot}^2 F^{+} = 4\pi R_{\odot}^2 \sigma T_{e}^4 , \; \frac{erg}{sec}
\end{equation}

H φαινόμενη λαμπρότητα του Ήλιου, $l_\odot$ που ονομάζεται {\it Ηλιακή Σταθερά}, είναι γνωστή απο αστρονομικές παρατηρήσεις. Η φαινόμενη λαμπρότητα ουσιαστικά ισούται με την φωτεινότητα $L_\odot$ διαμοιρασμένη σε μια επιφάνεια σφαίρας με ακτίνα την απόσταση Γης-Ήλιου.

\begin{equation}\label{eq:Brightness2}
  L_\odot = 4\pi (1 AU)^2 l_\odot,  \; \frac{erg}{sec}
\end{equation}

Απο τις \eqref{eq:SunsTotalPosFLux}, \eqref{eq:Brightness} και \eqref{eq:Brightness2} προκύπτει:

\begin{align}
  T_{e}^4 = \frac{4\pi (1 AU)^2 l_\odot}{4\pi R_{\odot}^2 \sigma} \nonumber \\
  T_{e}^4 = \frac{(1 AU)^2 l_\odot}{R_{\odot}^2 \sigma} \nonumber \\
  T_{e} = 5777.62 K \label{eq:SunsTemp}
\end{align}

Απο τον νόμο του {\en Wien}, \eqref{eq:WienLaw}, προκύπτει ότι το μέγιστο της εκπεμπόμενης ηλιακής ακτινοβολίας εντοπίζεται στα $\lambda_{max}=0.5016$ μ$m$, δηλαδή στην περιοχή του {\it Ορατού} τμήματος του ηλεκτρομαγνητικού φάσματος. Τελικά απο τις σχέσεις \eqref{eq:SunsTotalPosFLux} και \eqref{eq:Brightness} προκύπτει η φωτεινότητα του Ήλιου είναι:

\begin{equation}\label{eq:Brightness3}
  L_\odot = 3.8412 \times 10^{33} \; \frac{erg}{sec}
\end{equation}


\section{Υπολογισμός της Θερμοκρασίας Ισορροπίας της Σκόνης του Δίσκου}

Για να υπολογίσουμε την {\it Θερμοκρασία Ισορροπίας}, $T_{dust}$ της σκόνης πρέπει να πρώτα να συνυπολογίσουμε τον ρυθμό θέρμανσης και ψύξης ($H = \frac{erg}{sec}$ και $C= \frac{erg}{sec}$ αντίστοιχα) του κάθε σωματιδίου. Ο ρυθμός θέρμανσης ενός σωματιδίου σκόνης ισοδύναμεί με την ενέργεια που απορροφάει το σωματίδιο, ανα μονάδα χρόνου, απο τα φωτόνια όλων των συχνοτήτων του φάσματος προερχόμενα απο όλες τις διευθύνσεις.

\begin{equation}\label{eq:HeatingRate}
H =  \int_{0}^{\infty} k_{\nu} \pi J_{\nu} d\nu 
\end{equation}

Στο σημείο αυτό υποθέτουμε ότι κάθε στερεό σωματίδιο δέχεται απευθείας την ηλιακή ακτινοβολία, χωρίς δηλαδή κάποιο άλλο σωματίδιο να βρισκεται μπροστά του και να αποκόπτει μέρος αυτής ({\en optically thin Disc}). Τελικά κάθε σωματίδιο σκόνης θερμαίνεται αποκλειστικά απο την ακτινοβολία του Ήλιου\footnote{Aγνοούμε την ενέργεια που μπορεί να απορροφήσει ένα σωματίδιο απο τα υπόλοιπα σωματίδια, καθώς όλα εκπέμπουν προς όλες τις κατευθύνσεις ($4\pi$)}.\\

Ο Ήλιος για ένα σωματίδiο απόστασης, $r$, υπόκειται σε στερέα γωνία $\pi \frac{R_{\odot}^2}{r^2}$ και η είδικη ένταση της ακτινοβολίας που δέχεται το σωματίδιο είναι $B_{\nu}(T_{e})$. Απο την σχέση \eqref{eq:MeanIntensity} η μέση ένταση που δέχεται απο όλες τις διευθύνσεις (εντός της στερεάς γωνίας, $\pi \frac{R_{\odot}^2}{r^2}$) είναι:

\begin{equation}\label{eq:ParticleMeanIntensity}
 J_{\nu}=\frac{\pi R_{\odot}^2}{4 \pi r^2} B_{\nu}(T_{e}), \; \frac{erg}{sec \; cm^2 \; Hz \; ster} 
\end{equation}

Τελικά απο τις \eqref{eq:HeatingRate} και \eqref{eq:ParticleMeanIntensity} προκύπτει:

\begin{equation}\label{eq:HeatingRate2}
H = \frac{4(\pi r_{d})^2}{4m} (\frac{R_{\odot}}{r})^2 \int_{0}^{\infty} Q_{\nu}B_{\nu}(T_{e}) d\nu  
\end{equation}

Ο ρυθμός ψύξης ενός σωματιδίου ισοδύναμεί με την ενέργεια που εκπέμπει το σωματίδιο, ανά μονάδα χρόνου, για τα φωτόνια όλων των συχνοτήτων του φάσματος προς όλες τις κατευθύνσεις:

\begin{equation}\label{eq:CoolingRate2}
C = \int_{0}^{\infty} k_{\nu} \pi B_{\nu}(T_{dust}) d\nu = \frac{4(\pi r_{d})^2}{m} \int_{0}^{\infty} Q_{\nu}^{emis} B_{\nu}(T_{dust}) d\nu  
\end{equation}

Υποθέτωντας ότι μετά απο κάποιο χρονικό διάστημα τα σωματίδια της σκόνης, θερμοκρασίας $T_{dust}$, βρίσκονται σε θερμοδυναμική ισορροπία θα πρέπει ο ρυθμός ψύξης, $C$, να ισούται με τον ρυθμό θέρμανσης, $H$. Η παραπάνω υπόθεση προέρχεται απο τον Νόμο του {\en Kirchhoff} και κατ' επέκταση συμπεραίνουμε ότι όση ακτινοβολία απορροφάται τόση εκπέμπεται άρα μπορούμε να αντικαταστήσουμε στον ρυθμό ψύξης την αποδοτικότητα εκπομπής, $Q_{\nu}^{emis}$ \footnote{Η φυσική σημασία της αποδοτικότητας εκπομπής είναι ανάλογη με αυτήν την αποδοτικότητας απορρόφησης, όπως ορίστηκε, για την διαδικασία όμως της εκπομπής του φωτός. Συνήθως συμβολίζονται $Q_{\nu}^{emis}$ και $Q_{\nu}^{abs}$ αντίστοιχα, αλλα στην εργασία χρησιμοποιούμε $Q_{\nu}^{emis}$ και $Q_{\nu}$, με την  αποδοτικότητα απορρόφησης, $Q_{\nu}$}. Τελικά απο την σχέση $H=C$ έχουμε:

\begin{equation}\label{eq:CalcTdust}
(\frac{R_{\odot}}{2r})^2 \int_{0}^{\infty} Q_{\nu} B_{\nu}(T_{e}) d\nu = \int_{0}^{\infty} Q_{\nu} B_{\nu}(T_{dust}) d\nu 
\end{equation}

Στο σημείο αυτό θα γίνει μια ποιοτική ανάλυση για την λύση των δύο ολοκλήρωμάτων.\\
Όσον αφορά το ολοκλήρωμα στο αριστερό κομμάτι γνωρίζουμε ότι ο Ήλιος δεν εκπέμπει αποτελεσματικά στα μεγάλα μήκη κύματος,($\lambda_{IR}$ έως $\lambda = \infty$) $\cdot$ επίσης η σκόνη δεν απορροφά καθόλου στα πολύ μικρά μήκη κύματος,($\lambda = 0$ έως $\lambda_{UV}$). Τέλος τα μήκη κύματος, τα οποία είναι σχετικά με την αποτελεσματική απορρόφηση,($\lambda_{UV}$ έως $\lambda_{IR}$), είμαστε στο όριο της γεωμετρικής οπτικής οπότε μπορούμε να υποθέσουμε $Q_{\nu}=1$.\\
Όσον αφορά το ολοκλήρωμα στο δεξί κομμάτι γνωρίζουμε ότι η σκόνη εκπέμπει αποτελεσματικά στα μεγάλα μήκη κύματος όπου βρισκόμαστε στο όριο {\en Rayleigh} και ο συντελεστής $Q_{\nu}$ δίνεται απο τη σχέση \eqref{eq:AbsorEfficiency}.
Έτσι απο τις \eqref{eq:CalcTdust}, \eqref{eq:TotalInteBlackBody}, \eqref{eq:Brightness} για το αριστερό μέλος της ισότητας και απο την λύση του δεξιού μέλους της εξίσωσης\cite[{\en Chap.~8, Sect.~2.2}]{krugel2002physics} προκύπτει:

\begin{equation}
\frac{L_\odot}{16 (\pi r_d)^2} = 1.47\times10^{-6} r_d T_{dust}^6, \; \text{για $\beta=2$ και οι μονάδες είναι $r$ και $r_d$ se $cm$} 
\end{equation}

Μετατρέποντας την απόσταση $r$ σε $AU$ και για $r_d = 0.5$μ$m =5\times10^{-5} cm$ τελικά προκύπτει:

\begin{equation}\label{eq:DustTemp}
T_{dust} = 377.54 r^{- \frac{1}3}, \; \text{για $\beta=2$} 
\end{equation}

Η εξίσωση \eqref{eq:DustTemp} δίνει την θερμοκρασία ισορροπίας των {\en test particles} και κατ' επέκταση των σωματιδίων της σκόνης συναρτήσει της απόστασης.

\newpage
Παρακάτω δίνεται θερμοκρασία ισορροπίας των {\en test particles} (και κατ' επέκταση των σωματιδίων της σκόνης) συναρτήσει της απόστασης στον δίσκο, αλλά και η μέση θερμοκρασία κάθε δακτυλίου $i, i=1,2,3,...,120$ του δίσκου:

\begin{figure}[h]
\centering
 \begin{subfigure}{0.48\textwidth}
  \includegraphics[width=\linewidth]{T_dust}
  \caption{Θερμοκρασία των {\en test particles} συναρτησει της απόστασης τους}\label{fig:TPTemp}
 \end{subfigure}\hfill
 \begin{subfigure}{0.48\textwidth}
  \centering
  \includegraphics[width=\linewidth]{T_dustRi}
  \caption{Μέση θερμοκρασία κάθε δακτυλίου συναρτησει της απόστασης του}\label{fig:DiscTemp}
 \end{subfigure}
\end{figure}

Στο διάγραμμα \ref{fig:TPTemp} εντοπίζουμε δύο περιοχές όπου δεν υπάρχουν {\en test particles}. Φυσικά οι περιοχές αυτές αντιστοιχούν στον κενό δακτυλίο που έχει δημιουργήσει ο Δίας καθώς περιφέρεται στην τροχία του. Ακόμα στο διάγραμμα \ref{fig:DiscTemp} πρέπει να σημειωθεί ότι τα πρώτα 5 σημεία αντιστοιχούν στους δακτυλίους μέσης απόστασης $r_i =i\times0,17 AU$ με $i=1,2,3,4,5$ στους οποίους δεν εμπεριέχονται σωματίδια. Το γεγονός γίνεται φανερό και απο το διαγράμμα \ref{fig:TPTemp}, όπου τα {\en test paticles} εντοπίζονται απο αποστάσεις $r \geq 1AU$. Τέλος η εξίσωση \eqref{eq:DustTemp} δίνει ικανοποιητικά αποτελέσματα για σωματίδια διαμέτρου μέχρι $~1$μ$m$.\\







